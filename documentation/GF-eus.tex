% postontzi egokiak hilezkorrak dira.

\documentclass[11pt,a4paper]{article}
\usepackage[margin=30pt,head=50pt,headsep=30pt,foot=50pt]{geometry}
\usepackage{fontspec}
\usepackage{xltxtra}
\usepackage{xspace}
\usepackage[english]{babel}
\usepackage{calc}
\usepackage[table]{xcolor}
\usepackage{verbatim}
\usepackage[normalem]{ulem}
\usepackage{multicol}
\usepackage[small,bf]{caption}
\usepackage{multirow}
\usepackage{natbib}
\usepackage{tikz-qtree}
\usepackage{tikz-dependency}
\usepackage{setspace}
\usepackage{fancyhdr}
\usepackage{expex} % another possibility
\usepackage{url}



\begin{document}

% principles

% interlingua
% abstract syntax and concrete syntax

% structure of the grammar



\section{Simple sentence structure}

% Mutil ak katu a dauka .

\begin{center}
\begin{minipage}{.4\textwidth}
\begin{tikzpicture}
\Tree [.PredVP [.DetCN [.UseN boy\_N ] ] [.ComplSlash [.SlashV2a have\_V2 ] [.DetCN [.UseN cat\_N ] ] ] ] 
\end{tikzpicture}
\end{minipage}
\begin{minipage}{.4\textwidth}
\begin{tikzpicture}
\tikzset{frontier/.style={distance from root=150pt}}
\Tree [.Cl [.NP [.CN [.N mutil ] [.Det [.Quant ak ] ] ] ]
           [.VP 
                [.NP [.CN [.N katu ] ] [.Det [.Quant a ] ] ]
                [.VPSlash [.V2 dauka ] ] 
           ] ] 
\end{tikzpicture}
\end{minipage}
\end{center}

\section{Adverbial phrases}

% Mutil a itsaso a ra ibiltzen da .

\begin{center}
\begin{minipage}{.4\textwidth}
\begin{tikzpicture}
\Tree [.PredVP [.DetCN [.UseN boy\_N ] ] [.AdvVP [.UseV walk\_V ] [.PrepNP to\_Prep [.DetCN [.UseN sea\_N ] ] ] ] ]
\end{tikzpicture}
\end{minipage}
\begin{minipage}{.4\textwidth}
\begin{tikzpicture}
\tikzset{frontier/.style={distance from root=200pt}}
\Tree [.Cl [.NP [.CN [.N mutil ] [.Det [.Quant a ] ] ] ]
           [.VP 
                [.Adv [.NP [.CN [.N itsaso ] [.Det [.Quant a ] ] ] ] [.Prep ra ] ]
                [.V ibiltzen da ]
           ] ]
\end{tikzpicture}
\end{minipage}
\end{center}



% Txakur a txiki a da .

% Ez dauzkagu katu ik .


\section{Adnominal phrases with \emph{-ko}}

% Bilbo ra ko errepide a berri a da .

\begin{center}
\begin{minipage}{.4\textwidth}
???
\end{minipage}
\begin{minipage}{.4\textwidth}
\begin{tikzpicture}
\tikzset{frontier/.style={distance from root=200pt}}
\Tree [.Cl [.NP [.CN 
                     [.AP 
                       [.Adv [.NP [.PN Bilbo ] ] [.Prep ra ] ] 
                       [.Attr ko ]  ]
                     [.CN [.N errepide ] ] 
                ] 
                [.Det [.Quant a ] ]
           ]
           [.VP [.Comp [.AP [.A berri ] [.Det [.Quant a ] ] ] ] 
                [.V da ]
           ] ]
\end{tikzpicture}
\end{minipage}


\end{center}

% Diru a apaiz a ren a da.


% Non dago Miren ?


\section{Simple interrogative clauses}

\begin{center}
\begin{minipage}{.4\textwidth}
% UttQS (UseQCl (TTAnt TPres ASimul) PPos (QuestIComp (CompIAdv where_IAdv) (UsePN mary_PN))) 
\Tree [.UttQS [.UseQCl [.QuestIComp [.CompIAdv where\_IAdv ] ] [.UsePN mary\_PN ] ] ]
\end{minipage}
\begin{minipage}{.4\textwidth}
\begin{tikzpicture}
\tikzset{frontier/.style={distance from root=200pt}}
% (Utt:10 (QS:9 (QCl:8 (IComp:5 (IAdv:4 where)) is (NP:7 (PN:6 Mary)))))
\Tree [.Utt [.QS [.QCl [.IComp [.IAdv non ] dago [.NP [.PN Miren ] ] ] ] ] ] 
\end{tikzpicture}
\end{minipage}


\end{center}

\section{Verbs, VPs and clauses}
   Verb has only two fields: participle, which inflects only in tense, and 
   valency, which includes
   a) one of the four possible argument structures, and
   b) which synthetic verb is used for the inflecting part (only varies for 2 of the 4 structures).
   This means that all V*, regardless of the kind of complement, are of the same linearisation type.
   
   Why do we only include a parameter instead of the actual verb table?
   \footnote{In the PMCFG, what would happen? Would they only point to the same place, or would they be duplicated? TODO find out, if this is an optimisation at all!}
   The straight-forward solution would be to equip intransitive Vs with one-dimensional verb table;
   transitive verbs with two-dimensional, and by the time a V2 is combined with an object NP, the object agreement is resolved,
   leaving the originally intransitive and originally transitive VPs both with an one-dimensional verb table, waiting just the subject agreement.
   However, the alternative solution is inadequate at capturing the phenomenon described in Section~\ref{foo}, 
   on negative polarity changing an indefinite object NP into partitive, and forcing the object agreement into singular.

\begin{quote}
   Garagardoak edaten ditut   `I drink (the) beers' \\
   Ez dut garagardorik edaten `I don't drink (the) beer/beers' \\
   \verb!*!Ez ditut garagardorik edaten `I don't drink (the) beers'
\end{quote}

   Since the direct object agreement cannot be resolved at the VP level -- polarity comes only at two stages up, from Cl to S,
   we would be forced to carry the extra layer all the way up to S, or at least change the table argument to Polarity instead of Agreement.
   Another reason is the RGL's heavy usage of Slash categories: VPSlash is a verb phrase that may be missing one or more of
   direct object, indirect object or an adverbial complement.
   To combat this, we would have to make all the verbs 3-dimensional with dummy parameters, in order for the VPSlash category to use them.

   This is our category for VP:

   \begin{verbatim}
   VerbPhrase : Type = {} ...
   \end{verbatim}

   The \verb!VPSlash -> VP! functions add arguments into the fields of the VP(Slash), and the \verb!V* -> VPSlash! functions indicate which hole is to be filled. The list of \verb!V* -> VPSlash! functions is the following:

   \begin{verbatim}
    SlashV2a : V2        -> VPSlash ;  -- love (it)
    Slash2V3 : V3  -> NP -> VPSlash ;  -- give it (to her)
    Slash3V3 : V3  -> NP -> VPSlash ;  -- give (it) to her
    SlashV2V : V2V -> VP -> VPSlash ;  -- beg (her) to go
    SlashV2S : V2S -> S  -> VPSlash ;  -- answer (to him) that it is good
    SlashV2Q : V2Q -> QS -> VPSlash ;  -- ask (him) who came
    SlashV2A : V2A -> AP -> VPSlash ;  -- paint (it) red
    SlashVV    : VV  -> VPSlash -> VPSlash ;       -- want to buy
    SlashV2VNP : V2V -> NP -> VPSlash -> VPSlash ; -- beg me to buy
    AdvVPSlash : VPSlash -> Adv -> VPSlash ;  -- use (it) here
    AdVVPSlash : AdV -> VPSlash -> VPSlash ;  -- always use (it)
    VPSlashPrep : VP -> Prep -> VPSlash ;  -- live in (it) 
   \end{verbatim}

And functions that fill VPSlashes:

\begin{verbatim}
Question.gf:    ComplSlashIP  : VPSlash -> IP -> QVP ;   -- buys what 
Sentence.gf:    SlashVP   : NP -> VPSlash -> ClSlash ;      -- (whom) he sees
Verb.gf:    ComplSlash : VPSlash -> NP -> VP ; -- love it
\end{verbatim}

Thus, a sentence like ``I drink beer in the sea'' can be formed of the following way:

\begin{verbatim}
1) SlashV2a drink:V ===> drink:VPSlash
2) ComplSlash drink:VPSlash beer:N  ===> drink_beer:VP
3) VPSlashPrep drink_beer:VP in:Prep ===> drink_beer_in:VPSlash
4) ComplSlash drink_beer_in:VPSlash the_sea_NP ===> drink_beer_in_the_sea:VP
\end{verbatim}

For Basque, these functions do not build strings; they just store the added complements,
along with their agreements, into the right fields of the VP and VPSlash.
The type of VPSlash is just an extension of VP, with a parameter that tells which field is missing, so that ComplSlash can fill the right one.
Matching the two happens with the parameter Missing = MissingDObj | MissingIObj | MissingAdv.

Once we have a VP, where the number of arguments matches the valency (actually we have no check for that; if you accidentally type the wrong thing, the next function just picks 3rd person singular by default for the argument you missed!),
we can proceed to the PredVP function: it adds the subject, but leaves tense and polarity still open.

The type of Cl is 

   \begin{verbatim}
   Clause : Type = {} ...
   Sentence : Type = {} ...
   \end{verbatim}

The function mkClause does all the heavy work: resolves the inflecting verb form, and picks the right agreement, and right word order.






\end{document}
