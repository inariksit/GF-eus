% postontzi egokiak hilezkorrak dira.

\documentclass[11pt,a4paper]{article}
\usepackage[margin=30pt,head=50pt,headsep=30pt,foot=50pt]{geometry}
\usepackage{fontspec}
\usepackage{xltxtra}
\usepackage{xspace}
\usepackage[english]{babel}
\usepackage{calc}
\usepackage[table]{xcolor}
\usepackage{verbatim}
\usepackage[normalem]{ulem}
\usepackage{multicol}
\usepackage[small,bf]{caption}
\usepackage{multirow}
\usepackage{natbib}
\usepackage{tikz-qtree}
\usepackage{tikz-dependency}
\usepackage{setspace}
\usepackage{fancyhdr}
\usepackage{expex} % another possibility
\usepackage{url}



\begin{document}

% principles

% interlingua
% abstract syntax and concrete syntax

% structure of the grammar



\section{Simple sentence structure}

% Mutil ak katu a dauka .

\begin{center}
\begin{minipage}{.4\textwidth}
\begin{tikzpicture}
\Tree [.PredVP [.DetCN [.UseN boy\_N ] ] [.ComplSlash [.SlashV2a have\_V2 ] [.DetCN [.UseN cat\_N ] ] ] ] 
\end{tikzpicture}
\end{minipage}
\begin{minipage}{.4\textwidth}
\begin{tikzpicture}
\tikzset{frontier/.style={distance from root=150pt}}
\Tree [.Cl [.NP [.CN [.N mutil ] [.Det [.Quant ak ] ] ] ]
           [.VP 
                [.NP [.CN [.N katu ] ] [.Det [.Quant a ] ] ]
                [.VPSlash [.V2 dauka ] ] 
           ] ] 
\end{tikzpicture}
\end{minipage}
\end{center}

\section{Adverbial phrases}

% Mutil a itsaso a ra ibiltzen da .

\begin{center}
\begin{minipage}{.4\textwidth}
\begin{tikzpicture}
\Tree [.PredVP [.DetCN [.UseN boy\_N ] ] [.AdvVP [.UseV walk\_V ] [.PrepNP to\_Prep [.DetCN [.UseN sea\_N ] ] ] ] ]
\end{tikzpicture}
\end{minipage}
\begin{minipage}{.4\textwidth}
\begin{tikzpicture}
\tikzset{frontier/.style={distance from root=200pt}}
\Tree [.Cl [.NP [.CN [.N mutil ] [.Det [.Quant a ] ] ] ]
           [.VP 
                [.Adv [.NP [.CN [.N itsaso ] [.Det [.Quant a ] ] ] ] [.Prep ra ] ]
                [.V ibiltzen da ]
           ] ]
\end{tikzpicture}
\end{minipage}
\end{center}



% Txakur a txiki a da .

% Ez dauzkagu katu ik .


\section{Adnominal phrases with \emph{-ko}}

% Bilbo ra ko errepide a berri a da .

\begin{center}
\begin{minipage}{.4\textwidth}
???
\end{minipage}
\begin{minipage}{.4\textwidth}
\begin{tikzpicture}
\tikzset{frontier/.style={distance from root=200pt}}
\Tree [.Cl [.NP [.CN 
                     [.AP 
                       [.Adv [.NP [.PN Bilbo ] ] [.Prep ra ] ] 
                       [.Attr ko ]  ]
                     [.CN [.N errepide ] ] 
                ] 
                [.Det [.Quant a ] ]
           ]
           [.VP [.Comp [.AP [.A berri ] [.Det [.Quant a ] ] ] ] 
                [.V da ]
           ] ]
\end{tikzpicture}
\end{minipage}


\end{center}

% Diru a apaiz a ren a da.


% Non dago Miren ?


\section{Simple interrogative clauses}

\begin{center}
\begin{minipage}{.4\textwidth}
% UttQS (UseQCl (TTAnt TPres ASimul) PPos (QuestIComp (CompIAdv where_IAdv) (UsePN mary_PN))) 
\Tree [.UttQS [.UseQCl [.QuestIComp [.CompIAdv where\_IAdv ] ] [.UsePN mary\_PN ] ] ]
\end{minipage}
\begin{minipage}{.4\textwidth}
\begin{tikzpicture}
\tikzset{frontier/.style={distance from root=200pt}}
% (Utt:10 (QS:9 (QCl:8 (IComp:5 (IAdv:4 where)) is (NP:7 (PN:6 Mary)))))
\Tree [.Utt [.QS [.QCl [.IComp [.IAdv non ] dago [.NP [.PN Miren ] ] ] ] ] ] 
\end{tikzpicture}
\end{minipage}


\end{center}



\end{document}
